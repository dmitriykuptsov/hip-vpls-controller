\chapter{Background}
In this section we are going to describe some background 
material. We are going to start with the cryptographic primitives
such as symmetric key encryption/decryption algorithms and 
then move on to the discussion of the Virtual Private LAN 
Services and what kind of problems they solve. We will then
conclude the discussion in this chapter with the basic information
on Host Identity Protocol as it is in the core of the solution which we
are discussing in this document. To make the background material
more or less complete we are going to touch alternative 
Transport Layer Security (we will here mention why this 
protocol is not used in the core of our architecture and 
is only used for the control-plane communications between the 
HIP-Switches and the HIP-controller). With these final words
we are going to conclude current chapter of this work.

\section{Cryptography}
Cryptography forms the bases for secure telecommunications 
nowadays. SSL, TLS, SSH, Tacacs+, IPSec, DKIM, DNSSec are 
only few well-known telecommunication protocols that use 
cryptography to prevent such well-known attacks as 
eavesdropping, tampering, denial of message origin, etc. 
Modern cryptography is based on the hard-core 
mathematics and non-trivial algorithms (such as random number 
generation, discrete logarithm problem, rings, fields, 
Euclidean algorithm, factorization of big numbers, etc.)

\subsection{Symmetric cryptography}
Symmetric key cryptography is just perfect for the data-plane traffic
as it offers high-processing times (when compare to asymmetric key
cryptography). As the name implies, symmetric key cryptography uses
the same secret key to encrypt and decrypt messages. On one hand it is
the main reason why these algorithms are so fast. On the other hand, 
and this is the main limitation of the type of cryptography: 
symmetric keys are hard to distribute and revoke without using more
sophisticated symmetric key schemes. 

As of today several symmetric key cryptography algorithms, such as
Advanced Encryption Standard (AES), Tripple DES (3DES), and Twofish
offer advantageous processing speed and sufficient security levels.
In our prototype implementation of HIP-VPLS we are using AES with 256 
bits keys to perform encryption and decryption of data-plane traffic. Moreover,
since NanoPI R2S - hardware that we employ to run our Software Defined Network
(SDN) code - has support for on-chip instructions to boost the 
encryption and decryption of arbitrary long message blocks. In other words
we do perform AES operations directly in the Central Processing Unit (CPU) 
of the tiny computer. We are going to devote a separate section on the 
implementation of the hardware accelerated AES encryption and decryption 
routines by the CPU.

\subsection{Asymmetric cryptography}
Asymmetric key cryptography as the name suggests uses two separate keys
to encrypt and decrypt the messages. Since the encryption uses big number 
exponentiations (such as RSA) and multiplications (such as EDDSA), as well as
modular arthihmetics, the performance of these types of algorithms is 
considerably worse when compared to symmetric cryptography algorithms. 

However, since one is allowed to expose public part of the key to anyone,
and since this key is only required to encrypt the message and only person
who holds the private part of the key (secret part of the key) can decrypt the 
message, efficient key distribution and revocation can be organized, at the cost
of extra CPU cycles. Moreover, by encrypting the message with the private key,
and then making decryption plausible only with a public key (exposed to everyone),
digital signature schemes can be implemented at no hassle.  

\subsection{Hash functions}
\subsection{Key exchange protocols}
\section{Virtual Private LAN Service}
\section{Host Identity Protocol}
